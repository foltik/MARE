\documentclass{article}
\usepackage[utf8]{inputenc}

\title{Self Modifying Code \\ \small Malware Analysis and Reverse Engineering Proposal, Spring 2020}
\author{Jack Foltz, Andrew Park, Brian Qi}
\date{February 2020}

\usepackage{natbib}
\usepackage{graphicx}

\begin{document}

\maketitle


\section{Problem Statement}
Self modifying code covers broad category of all types of code that
are modified in some way, before running on the target platform. This
definition includes code that modifies machine code directly or
modifies code during runtime. Often, this technique is employed for
optimization purposes: code can replace conditional branch statements
with unconditional branch statements, or patch dynamic library loads.
However, we are primarily intetrested in its security applications, in
which code modifies itself at runtime in order to evade different
methods of malware detection.

Basic malware typically has some sort of identifiable signature, such
as a unique pattern of instructions present in the binary discovered
through static analysis. By using self modifying code, static analysis
can be rendered completely useless, since only a small bit of generic
code that modifies itself is present in the binary. While traditional
malware authors have used packing to encrypt the payload of a virus to
evade detection, the unpacker itself must be left exposed. However, it
is possible to even obsucure this unpacker part as well
\cite{cai2007certified}.

We seek to create a piece of software that modifies itself at runtime
in order to evade such detection vectors, testing it out with both
self made static analysis tools, and real world anti-malware software.


\section{Motivation}
Effective detection and mitigation of malware is an important security
goal for modern operating systems and anti-malware software.

Understanding the methods by which malware can obscure itself through
usage of self modifying code may lead to new insights into the
characteristics of a good solution for detecting such malware, or how
this technique could be even further enhanced.



\bibliographystyle{plain}
\bibliography{references}
\end{document}
