% Created 2020-02-10 Mon 10:44
% Intended LaTeX compiler: pdflatex
\documentclass[11pt]{article}
\usepackage[utf8]{inputenc}
\usepackage[T1]{fontenc}
\usepackage{graphicx}
\usepackage{grffile}
\usepackage{longtable}
\usepackage{wrapfig}
\usepackage{rotating}
\usepackage[normalem]{ulem}
\usepackage{amsmath}
\usepackage{textcomp}
\usepackage{amssymb}
\usepackage{capt-of}
\usepackage{hyperref}
\usepackage{minted}
\date{\today}
\title{HW1}
\hypersetup{
 pdfauthor={},
 pdftitle={HW1},
 pdfkeywords={},
 pdfsubject={},
 pdfcreator={Emacs 27.0.50 (Org mode 9.1.9)}, 
 pdflang={English}}
\begin{document}

\maketitle
\tableofcontents

\href{https://docs.google.com/document/d/1aNu32\_XUJ-QjYssRXdDGC0mFVrMKvvZ5pdPrUU9PejU/edit}{Questions}

\section{Lab 1-1}
\label{sec:org40add04}
\subsection{Question 1}
\label{sec:orga6e9e0e}
Yes, 40/71 antiviruses report the executable as a virus, and 32/69
report the DLL as a virus.
\subsection{Question 2}
\label{sec:org7e72cb3}
The Time Date Stamp in \texttt{IMAGE\_FILE\_HEADER} reports compile times of
\texttt{2010/12/19 Sun 16:16:19 UTC}, and \texttt{2010/12/19 Sun 16:16:38 UTC} for
the executable and DLL respectively.
\subsection{Question 3}
\label{sec:orgc90671f}
No, but dependency walker reports only two DLL imports for the
executable, which likely it loads the other DLL in the folder with it.
Neither appear to be packed.
\subsection{Question 4}
\label{sec:org30a9a4e}
The executable appears to import various functions related to reading,
writing, and copying files.

\begin{center}
\includegraphics[width=.9\linewidth]{./lab1-1-imports.png}
\end{center}

The DLL appears to import functions related to creating mutexes and
processes, and sending/receiving messages over network sockets.

\begin{center}
\includegraphics[width=.9\linewidth]{./lab1-1-dll-imports.png}
\end{center}
\subsection{Question 5}
\label{sec:orge7aa3f2}
We could also check for any static strings in both files, which could
tip us off on what specific things each file does.

The executable contains a reference to the file name of the DLL file,
along with \texttt{"WARNING\_THIS\_WILL\_DESTROY\_YOUR\_MACHINE"} , and a
suspicious string \texttt{"C:\textbackslash{}windows\textbackslash{}system32\textbackslash{}kerne132.dll"}, which is
extremely similar to the system DLL kernel32.dll.

\begin{center}
\includegraphics[width=.9\linewidth]{./lab1-1-exe-strings.png}
\end{center}

The DLL has strings of an IP address, and a few strings that sound
like messages sent to that IP: \texttt{"execute"} , \texttt{"hello"} , and \texttt{"sleep"}.

\begin{center}
\includegraphics[width=.9\linewidth]{./lab1-1-dll-strings.png}
\end{center}
\subsection{Question 6}
\label{sec:org447e55c}
We could use wireshark to capture inbound or outbound network traffic,
checking specifically for the IP address we saw in the strings.
\subsection{Question 7}
\label{sec:orga1c94de}
I would guess that the executable drops the DLL into
\texttt{C:\textbackslash{}windows\textbackslash{}system32\textbackslash{}kerne132.dll} and then executes it.

From host based indicators I thought it might open a socket, connect
to an IP, and send some sort of hello.

\begin{center}
\includegraphics[width=.9\linewidth]{./lab1-1-disas.png}
\end{center}

After static analysis, it looks like it receives executable files and
will spawn processes with code sent over the network by the server.
\section{Lab 1-2}
\label{sec:orgddc45df}
\subsection{Question 1}
\label{sec:org3cefb95}
Yes, 39/63 antiviruses report it as a virus.
\subsection{Question 2}
\label{sec:orgaee3817}
Using PEiD, the executable is reported as "Nothing Found*", but the
executable has no normal \texttt{.text} or any other sections, just \texttt{UPX0},
\texttt{UPX1}, \texttt{UPX2}, which makes me think it is packed with \texttt{UPX}.

\begin{center}
\includegraphics[width=.9\linewidth]{./lab1-2-peid.png}
\end{center}

After downloading and using the UPX tool on the file, we can see that
it was successfully unpacked and we can see the real sections again.

\begin{center}
\includegraphics[width=.9\linewidth]{./lab1-2-unpack.png}
\end{center}
\subsection{Question 3}
\label{sec:orgb6bd06f}
It appears to import various functions related to opening internet
URLs, mutexes, and creating services. This tells us that this malware
probably downloads a malicious file from the internet and creats a
persistent service that runs it.

\begin{center}
\includegraphics[width=.9\linewidth]{./lab1-2-imports.png}
\end{center}
\subsection{Question 4}
\label{sec:orge8890a8}
We could check the system services for the infected service that the
malware creates, or determine the sites that the malware makes network
requests to by further analysis, and monitor traffic to these sites.
\section{Lab 1-3}
\label{sec:org55134e6}
\subsection{Question 1}
\label{sec:org5fde957}
Yes, 61/71 antiviruses report it as a virus.
\subsection{Question 2}
\label{sec:orga0812bf}
Using PEiD, the executable is reported as being packed with FSG 1.0.

\begin{center}
\includegraphics[width=.9\linewidth]{./lab1-3-peid.png}
\end{center}

After using OllyDBG to find the original entry point, I was able to
dump the packed executable and reconstruct the imports table with
OllyDump.

\begin{center}
\includegraphics[width=.9\linewidth]{./lab1-3-dbg.png}
\end{center}

\begin{center}
\includegraphics[width=.9\linewidth]{./lab1-3-unpacked.png}
\end{center}

Looking at the disassembly, we can see a call to \texttt{CoCreateInstance},
which creats an instance of a COM object.

\begin{center}
\includegraphics[width=.9\linewidth]{./lab1-3-cocreate.png}
\end{center}

We can determine which COM object it is instantiating via the \texttt{riid}
parameter, which after looking at it in the data segment, appears to
be \texttt{D30C1661-CDAF-11D0-A83E-00C04FC9E26E}.

\begin{center}
\includegraphics[width=.9\linewidth]{./lab1-3-riid.png}
\end{center}

Searching the registry for this value, we find that the COM object is
\texttt{IWebBrowser2}.

\begin{center}
\includegraphics[width=.9\linewidth]{./lab1-3-reg.png}
\end{center}

We can then add a structure that contains the vtable for this COM
object to see what methods the malware calls:

\begin{center}
\includegraphics[width=.9\linewidth]{./lab1-3-struct.png}
\end{center}

Now we can update the offset to \texttt{edx} to be an offset into this
struct, and we can see that the malware calls the \texttt{Navigate} function,
which opens a web browser to the passed URL.

\begin{center}
\includegraphics[width=.9\linewidth]{./lab1-3-resolve.png}
\end{center}
\subsection{Question 3}
\label{sec:orgb028669}
It appears to import functions related to manipulating COM objects, so
it likely calls out to some other COM api.
\subsection{Question 4}
\label{sec:org6b0e262}
While the malware doesn't appear to do much, we could use Wireshark or
other network monitoring tools to watch for internet traffic to the
URL we found.
\section{Lab 1-4}
\label{sec:org75c0b34}
\subsection{Question 1}
\label{sec:org28dc5c8}
Yes, 55/68 antiviruses report it as a virus.
\subsection{Question 2}
\label{sec:orgc9357f0}
Using PEiD, the executable doesn't appear to be packed.

\begin{center}
\includegraphics[width=.9\linewidth]{./lab1-4-peid.png}
\end{center}
\subsection{Question 3}
\label{sec:orgc985e00}
The Time Date Stamp in \texttt{IMAGE\_FILE\_HEADER} reports a compile time of
\texttt{2019/08/30 Fri 22:26:59 UTC}.
\subsection{Question 4}
\label{sec:orga75e989}
It appears to import various functions related to reading attached
resources, loading libraries, creating thread in remote processes, and
writing files.

These routines seem typical of a malware that injects code into
another process.

\begin{center}
\includegraphics[width=.9\linewidth]{./lab1-4-imports.png}
\end{center}
\subsection{Question 5}
\label{sec:org753f7a6}
We could use Wireshark or other network monitoring tools to watch for
internet traffic, checking specifically for the URL present in the
strings of the attached resources.
\subsection{Question 6}
\label{sec:org54f84bf}
Using Resource Hacker, we can see the file contains one resource which
is also an executable (the data starts with the two bytes \texttt{MZ}).

\begin{center}
\includegraphics[width=.9\linewidth]{./lab1-4-resource.png}
\end{center}

Looking at the imports and strings, this appears to be the part of the
malware that downloads and executes a file from the internet.

\begin{center}
\includegraphics[width=.9\linewidth]{./lab1-4-resource-imports.png}
\end{center}

\begin{center}
\includegraphics[width=.9\linewidth]{./lab1-4-resource-strings.png}
\end{center}
\end{document}