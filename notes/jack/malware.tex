% Created 2020-04-26 Sun 16:37
% Intended LaTeX compiler: pdflatex
\documentclass[11pt]{article}
\usepackage[utf8]{inputenc}
\usepackage[T1]{fontenc}
\usepackage{graphicx}
\usepackage{grffile}
\usepackage{longtable}
\usepackage{wrapfig}
\usepackage{rotating}
\usepackage[normalem]{ulem}
\usepackage{amsmath}
\usepackage{textcomp}
\usepackage{amssymb}
\usepackage{capt-of}
\usepackage{hyperref}
\usepackage{minted}
\date{\today}
\title{HW7}
\hypersetup{
 pdfauthor={},
 pdftitle={HW7},
 pdfkeywords={},
 pdfsubject={},
 pdfcreator={Emacs 27.0.50 (Org mode 9.1.9)}, 
 pdflang={English}}
\begin{document}

\maketitle
\tableofcontents

\href{https://docs.google.com/document/d/1D7sqxEZtkl7xTF-tSeeWKc-5s-MkAScSbe2RgpyrOZE/edit}{Questions}
\section{Lab 9-1}
\label{sec:org70ffdab}
\subsection{Question 1}
\label{sec:org4e97f33}
We can see that the malware does a check on one of the command line
parameters used to launch it.

\begin{center}
\includegraphics[width=.9\linewidth]{./lab9-1-check.png}
\end{center}

If this function returns \texttt{1}, malware continues on, otherwise it
deletes itself.

After that, if the \texttt{-in} command line paramter is present, the malware
will install itself as a service either with the executable name or
another specified name.

\begin{center}
\includegraphics[width=.9\linewidth]{./lab9-1-install.png}
\end{center}
\subsection{Question 2}
\label{sec:org787c780}
The command line parameters appear to be:

\begin{itemize}
\item \texttt{-in} to install the service
\item \texttt{-re} to remove the service
\item \texttt{-c} to update the registry key
\item \texttt{-cc} to read the registry key
\end{itemize}

Reversing the \texttt{verify} routine, we can see that the required parameter
is \texttt{"abcd"}.

\begin{center}
\includegraphics[width=.9\linewidth]{./lab9-1-pass.png}
\end{center}
\subsection{Question 3}
\label{sec:org6239c7d}
We could simply patch the conditional jump after the verify routine to
invert it and continue only if the parameter was wrong.
\subsection{Question 4}
\label{sec:orgb390e06}
The malware creates a key in
\texttt{HKLM\textbackslash{}SOFTWARE\textbackslash{}Microsoft\textbackslash{}XPS\textbackslash{}Configuration}, drops itself in
\texttt{system32}, and creates a service with the dropped executable.
\subsection{Question 5}
\label{sec:orga2f801f}
Tracing the execution, we can see that if no parameters are called
then the malware gets commands from the network.

These include:
\begin{itemize}
\item \texttt{SLEEP}
\item \texttt{UPLOAD} a local file
\item \texttt{DOWNLOAD} a remote file
\item \texttt{CMD} execute a command
\item \texttt{NOTHING}
\end{itemize}

\begin{center}
\includegraphics[width=.9\linewidth]{./lab9-1-cmd.png}
\end{center}
\subsection{Question 6}
\label{sec:org1233ad7}
We can see that the malware makes requests to the URL specified in the
registry key to get its commands, which is by default \texttt{www.practicalmalwareanalysis.com}.
\section{Lab 9-2}
\label{sec:org656b687}
\subsection{Question 1}
\label{sec:org7c28a81}
There appear to be a lot of random error messages, and some imports.

\begin{center}
\includegraphics[width=.9\linewidth]{./lab9-2-strings.png}
\end{center}
\subsection{Question 2}
\label{sec:orge42f966}
It appears to do nothing.
\subsection{Question 3}
\label{sec:org3a31948}
Before the malware executes it copies two literal strings into a stack variable.

\begin{center}
\includegraphics[width=.9\linewidth]{./lab9-2-literals.png}
\end{center}

Afterwards, it checks if the executable name is \texttt{"ocl.exe"}, otherwise
exits.

\begin{center}
\includegraphics[width=.9\linewidth]{./lab9-2-cmp.png}
\end{center}
\subsection{Question 4}
\label{sec:orga179051}
A null terminated string is being copied into a stack variable
character by character.
\subsection{Question 5}
\label{sec:org29229ee}
One of the stack strings from before, \texttt{"1qaz2wsx3edc"} is being passed
along with another block of seemingly random data copied from a static
location at the start of \texttt{main}.

\begin{center}
\includegraphics[width=.9\linewidth]{./lab9-2-args.png}
\end{center}
\subsection{Question 6}
\label{sec:org3726b8b}
Setting a breakpoint right before \texttt{gethostbyname}, we can see that the
domain used is \texttt{practicalmalwareanalysis.com}.

\begin{center}
\includegraphics[width=.9\linewidth]{./lab9-2-host.png}
\end{center}
\subsection{Question 7}
\label{sec:org979d217}
The string \texttt{"1qaz2wsx3edc"} is used as the xor key to decode the
random static bytes.

\begin{center}
\includegraphics[width=.9\linewidth]{./lab9-2-xor.png}
\end{center}
\subsection{Question 8}
\label{sec:orge3e6b3a}
\texttt{hStdInput}, \texttt{hStdOutput}, and \texttt{hStdError} are all redirected over the
socket, which has the effect of opening a reverse shell to the command
and control server.

\begin{center}
\includegraphics[width=.9\linewidth]{./lab9-2-proc.png}
\end{center}
\section{Lab 9-3}
\label{sec:orgd301bba}
I ended up starting the lab close to the time it was due, so I didn't
get to finish Lab 9-3.
\end{document}