% Created 2020-04-05 Sun 12:29
% Intended LaTeX compiler: pdflatex
\documentclass[11pt]{article}
\usepackage[utf8]{inputenc}
\usepackage[T1]{fontenc}
\usepackage{graphicx}
\usepackage{grffile}
\usepackage{longtable}
\usepackage{wrapfig}
\usepackage{rotating}
\usepackage[normalem]{ulem}
\usepackage{amsmath}
\usepackage{textcomp}
\usepackage{amssymb}
\usepackage{capt-of}
\usepackage{hyperref}
\usepackage{minted}
\date{\today}
\title{HW4}
\hypersetup{
 pdfauthor={},
 pdftitle={HW4},
 pdfkeywords={},
 pdfsubject={},
 pdfcreator={Emacs 27.0.50 (Org mode 9.1.9)}, 
 pdflang={English}}
\begin{document}

\maketitle
\tableofcontents

\href{https://docs.google.com/document/d/1lYViOnBkWpGtFUB4LzteoEvrtec5kBMcOVtUTJ0GxW4/edit}{Questions}
\section{Lab 6-1}
\label{sec:org06a2d99}
\subsection{Question 1}
\label{sec:org283b601}
Checks to see if there is an internet connection present, and printing
a message accordingly.

\begin{center}
\includegraphics[width=.9\linewidth]{./lab6-1-mainsub.png}
\end{center}
\subsection{Question 2}
\label{sec:org5066640}
It appears to be some sort of printing function.
\subsection{Question 3}
\label{sec:org312ede5}
It returns with exit code 1 if there is no internet connection,
else 0.
\section{Lab 6-2}
\label{sec:org0e67210}
\subsection{Question 1}
\label{sec:org5e6147b}
Same as Lab 6-1.

\begin{center}
\includegraphics[width=.9\linewidth]{./lab6-2-mainsub.png}
\end{center}
\subsection{Question 2}
\label{sec:org470bc5c}
Same as Lab 6-1.
\subsection{Question 3}
\label{sec:org7fa77b9}
It reads the contents of the page
\url{http://www.practicalmalwareanalysis.com/cc.htm} into a buffer.
\subsection{Question 4}
\label{sec:org484141d}
Looks like an unrolled loop, looking for a string beginning with \texttt{<!-{}-},
which is the start of an HTML comment.

\begin{center}
\includegraphics[width=.9\linewidth]{./lab6-2-loop.png}
\end{center}
\subsection{Question 5}
\label{sec:org482faa5}
The program checks the internet connection, and if connected makes a
request to \url{http://www.practicalmalwareanalysis.com/cc.htm}.
\subsection{Question 6}
\label{sec:orga605e83}
It checks the internet connection, and if so prints out a byte as a
"command" from an HTML comment on a webpage, then sleeps and exits.
\section{Lab 6-3}
\label{sec:orgcb6b7dc}
\subsection{Question 1}
\label{sec:org2d717ac}
There's a new subroutine that does something with the command instead
of just exiting.

\begin{center}
\includegraphics[width=.9\linewidth]{./lab6-3-sub.png}
\end{center}
\subsection{Question 2}
\label{sec:org19abf77}
A filename and a buffer. In this case it's called with the path of the
running program, and the command.
\subsection{Question 3}
\label{sec:org57ed1a2}
Looks like a jump table based on the command.

\begin{center}
\includegraphics[width=.9\linewidth]{./lab6-3-jumps.png}
\end{center}
\subsection{Question 4}
\label{sec:orgc98395e}
It can do 5 different things depending on the command, either create a
directory \texttt{C:\textbackslash{}Temp}, copy itself to \texttt{C:\textbackslash{}Temp\textbackslash{}cc.exe}, delete
\texttt{C:\textbackslash{}Temp\textbackslash{}cc.exe}, add \texttt{C:\textbackslash{}Temp\textbackslash{}cc.exe} to the startup registry key, or
sleep and exit.
\subsection{Question 5}
\label{sec:org91519e1}
It can create the file \texttt{C:\textbackslash{}Temp\textbackslash{}cc.exe} or the registry key
\texttt{Software\textbackslash{}Microsoft\textbackslash{}Windows\textbackslash{}CurrentVersion\textbackslash{}Run\textbackslash{}Malware}.
\subsection{Question 6}
\label{sec:orgddc8f56}
The program makes sure there is an internet connection, then reads a
command from a command and control server, then does some various
things based on the command.
\section{Lab 6-4}
\label{sec:orgd790e3c}
\subsection{Question 1}
\label{sec:org3c55329}
The command parsing function has a new argument.
\subsection{Question 2}
\label{sec:org20ee8a4}
A loop has been added around the main body.

\begin{center}
\includegraphics[width=.9\linewidth]{./lab6-4-loop.png}
\end{center}
\subsection{Question 3}
\label{sec:orgbd5b4b5}
It now uses the loop counter in the user agent used to make the
request to the webpage.
\subsection{Question 4}
\label{sec:org238c5e1}
The main loop runs 1440 times, each loop sleeping 60 seconds plus any
network request time, so for around 1 day.
\subsection{Question 5}
\label{sec:orgea8b759}
The user agent is different this time.
\subsection{Question 6}
\label{sec:orge0deb04}
Same as the last one, except it now executes commands repeatedly
instead of just once.
\end{document}