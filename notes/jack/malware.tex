% Created 2020-04-19 Sun 14:08
% Intended LaTeX compiler: pdflatex
\documentclass[11pt]{article}
\usepackage[utf8]{inputenc}
\usepackage[T1]{fontenc}
\usepackage{graphicx}
\usepackage{grffile}
\usepackage{longtable}
\usepackage{wrapfig}
\usepackage{rotating}
\usepackage[normalem]{ulem}
\usepackage{amsmath}
\usepackage{textcomp}
\usepackage{amssymb}
\usepackage{capt-of}
\usepackage{hyperref}
\usepackage{minted}
\date{\today}
\title{HW5}
\hypersetup{
 pdfauthor={},
 pdftitle={HW5},
 pdfkeywords={},
 pdfsubject={},
 pdfcreator={Emacs 27.0.50 (Org mode 9.1.9)}, 
 pdflang={English}}
\begin{document}

\maketitle
\tableofcontents

\href{https://docs.google.com/document/d/1-ne6wM4j62DeNSo47u5rZV3XYM8Ev\_LplBILcEMCsNo/edit}{Questions}
\section{Lab 7-1}
\label{sec:orga90c2da}
\subsection{Question 1}
\label{sec:org6c81abf}
The malware creates a service called \texttt{MalService}.
\begin{center}
\includegraphics[width=.9\linewidth]{./lab7-1-svc.png}
\end{center}
\subsection{Question 2}
\label{sec:orgb31c276}
It prevents multiple instances of the malware from running at the same
time.
\subsection{Question 3}
\label{sec:org84b0371}
The mutex and the service.
\subsection{Question 4}
\label{sec:orgd1b2e60}
The malware opens a URL in internet explorer with a predefined user agent.
\begin{center}
\includegraphics[width=.9\linewidth]{./lab7-1-net.png}
\end{center}
\subsection{Question 5}
\label{sec:orgc1a5f0f}
The malware waits until a certain date, then creates 20 threads that
make requests to \texttt{practicalmalwareanalysis.com} in a loop.
\subsection{Question 6}
\label{sec:org884ce1e}
The program waits until the target date, then sends requests forever.
\section{Lab 7-2}
\label{sec:org09d48a9}
\subsection{Question 1}
\label{sec:org8c3c389}
As far as I can tell, it doesn't.
\subsection{Question 2}
\label{sec:org0424741}
It uses the same method from a previous lab to display the webpage
\texttt{malwareanalysisbook.com/ad.html}.
\subsection{Question 3}
\label{sec:org8d42f82}
Right after the page is opened.
\section{Lab 7-3}
\label{sec:org2473e02}
\subsection{Question 1}
\label{sec:orgc66a0e0}
The malware maps copies of both the malicious DLL and
\texttt{System32\textbackslash{}Kernel32.dll}, makes a bunch of weird patches, to the mapped
files, then copies it to \texttt{System32\textbackslash{}kerne132.dll}. It then calls a
function with the parameter \texttt{C:\textbackslash{}*}. This function walks the directory
calling itself recursively on all subfolders, and calling another
function on any \texttt{.exe} files found. This next function maps the file
and does a string search for \texttt{Kernel32.dll}, replacing it with the
malicious \texttt{kerne132.dll}, which has the effect of overwriting the
import table so the malicious DLL is loaded by every executable
infected.

\begin{center}
\includegraphics[width=.9\linewidth]{./lab7-3-rewrite.png}
\end{center}
\subsection{Question 2}
\label{sec:org6d27dd1}
The malicious DLL resides in \texttt{System32\textbackslash{}kerne132.dll}, and creates a
mutex called \texttt{SADHUHF},
\subsection{Question 3}
\label{sec:org245f3c1}
It infects every executable on the system with an import of a
malicious DLL, which once running opens a socket and reads commands
from \texttt{127.26.152.13}, which includes starting arbitrary processes.

\begin{center}
\includegraphics[width=.9\linewidth]{./lab7-3-control.png}
\end{center}
\subsection{Question 4}
\label{sec:org3eecc59}
You would have to fix the import table of every single affected
executable. Or\ldots{} a quick temporary fix would be to replace the
malicious \texttt{kerne132.dll} with a copy of the original \texttt{Kernel32.dll}.
\end{document}