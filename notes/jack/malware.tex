% Created 2020-04-20 Mon 11:36
% Intended LaTeX compiler: pdflatex
\documentclass[11pt]{article}
\usepackage[utf8]{inputenc}
\usepackage[T1]{fontenc}
\usepackage{graphicx}
\usepackage{grffile}
\usepackage{longtable}
\usepackage{wrapfig}
\usepackage{rotating}
\usepackage[normalem]{ulem}
\usepackage{amsmath}
\usepackage{textcomp}
\usepackage{amssymb}
\usepackage{capt-of}
\usepackage{hyperref}
\usepackage{minted}
\date{\today}
\title{HW6}
\hypersetup{
 pdfauthor={},
 pdftitle={HW6},
 pdfkeywords={},
 pdfsubject={},
 pdfcreator={Emacs 27.0.50 (Org mode 9.1.9)}, 
 pdflang={English}}
\begin{document}

\maketitle
\tableofcontents

\section{Phase 1}
\label{sec:org63362a4}
This phase simply does only simple string compare.
\begin{center}
\includegraphics[width=.9\linewidth]{./bomb-1.png}
\end{center}
\section{Phase 2}
\label{sec:org066b306}
This phase reads in six numbers and checks the values.

We can easily see that the first number has to be one. The following
code checks in a loop that the next number is equal to its index plus
one, multiplied by the last index.

That means that the second number has to be \texttt{2 * 1 = 2}, then \texttt{3 * 2 =
6}, \texttt{4 * 6 = 24}, etc.

This yields the solution \texttt{1 2 6 24 120 720}.

\begin{center}
\includegraphics[width=.9\linewidth]{./bomb-2.png}
\end{center}
\section{Phase 3}
\label{sec:org1703431}
This phase reads in two numbers and a character.

The first number is indexed into a jump table with 8 cases, exploding
if out of range.

Each target of the jump table compares the second number with a
required value, and loads a required value for the char to be compared
with.

\begin{center}
\includegraphics[width=.9\linewidth]{./bomb-3.png}
\end{center}


We can just pick the first one, from which we can see the solution is
\texttt{0 q 777}.

\begin{center}
\includegraphics[width=.9\linewidth]{./bomb-3-case.png}
\end{center}
\section{Phase 4}
\label{sec:org38f0118}
This phase reads in a single number that must be greater than zero,
calls a function, and explodes unless the return value is \texttt{55}.

\begin{center}
\includegraphics[width=.9\linewidth]{./bomb-4.png}
\end{center}

This function can be summarized as:

\begin{minted}[]{python}
def func4(n):
    if n <= 1:
        return 1
    else:
        return func4(n - 1) + func4(n - 2)
\end{minted}

Which clearly computes fibonacci numbers.

Thus, the required input to get \texttt{55} is \texttt{9}.

\begin{center}
\includegraphics[width=.9\linewidth]{./bomb-4-func.png}
\end{center}
\section{Phase 5}
\label{sec:orgce04ad2}
This phase reads in a string of 6 characters.

It translates the input string into a new buffer, comparing it with
the string \texttt{"giants"}.

The function takes the lower 4 bits of each input character, indexing
it into a table to get the resulting character.

We can inspect the table to figure out that the required byte sequence
get \texttt{"giants"} is \texttt{1111, 0000, 0101, 1011, 1101, 0001}. Cross
referencing this with a binary ascii table, one possible input string
that produces this sequence is \texttt{"O@EKMA"}.

\begin{center}
\includegraphics[width=.9\linewidth]{./bomb-5.png}
\end{center}
\section{Phase 6}
\label{sec:orga696134}
This phase again reads in six numbers.

The function first makes sure all numbers are <= 6.

\begin{center}
\includegraphics[width=.9\linewidth]{./bomb-6-1.png}
\end{center}

Then it starts a nested loop, comparing each number with every other
number. The bomb explodes if any two are the same.

\begin{center}
\includegraphics[width=.9\linewidth]{./bomb-6-2.png}
\end{center}


We see that structures in the data segment are indexed based on the
numbers. Each successive compare must be less than the previous.
Inspecting these values, we clearly see a linked list of values, and
sorting them in decreasing order yields the solution \texttt{4 2 6 3 1 5}.

\begin{center}
\includegraphics[width=.9\linewidth]{./bomb-6-3.png}
\end{center}
\section{Secret Phase}
\label{sec:org4ed7edc}
The bomb also contains a secret phase. It is called by the
\texttt{phase\_defused} method after phase 4 is solved, and scans for the
string \texttt{"austinpowers"} in addition to the integer solution.

\begin{center}
\includegraphics[width=.9\linewidth]{./bomb-secret-1.png}
\end{center}

The phase reads in another integer, making sure it is less than or
equal to \texttt{1001}.

It then calls a new function with another data segment structure, and
explodes unless the result is 7.

\begin{center}
\includegraphics[width=.9\linewidth]{./bomb-secret-2.png}
\end{center}


Looking at this structure, it's clearly a binary tree.

\begin{center}
\includegraphics[width=.9\linewidth]{./bomb-secret-3.png}
\end{center}

If the passed tree node's value is zero, the function returns -1. If
the node's value is less than the input, it returns twice the result
of recursing with the left child, and the same input. Otherwise, it
returns twice the result plus one of recursing with the right child,
and the same input.

In order to solve this, we must find a path down the tree where the
result is multiplied and added to get 7.

Tracing out the call structure, we find that the input \texttt{1001} is
required, defusing the secret phase.

\begin{center}
\includegraphics[width=.9\linewidth]{./bomb-secret-4.png}
\end{center}
\end{document}