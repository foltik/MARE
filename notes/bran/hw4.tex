% Created 2020-04-04 Sat 15:38
\documentclass[11pt]{article}
\usepackage[utf8]{inputenc}
\usepackage[T1]{fontenc}
\usepackage{fixltx2e}
\usepackage{graphicx}
\usepackage{longtable}
\usepackage{float}
\usepackage{wrapfig}
\usepackage{rotating}
\usepackage[normalem]{ulem}
\usepackage{amsmath}
\usepackage{textcomp}
\usepackage{marvosym}
\usepackage{wasysym}
\usepackage{amssymb}
\usepackage{hyperref}
\tolerance=1000
\author{dj$_{\text{dvorak}}$}
\date{\today}
\title{hw4}
\hypersetup{
  pdfkeywords={},
  pdfsubject={},
  pdfcreator={Emacs 25.2.2 (Org mode 8.2.10)}}
\begin{document}

\maketitle
\tableofcontents

\section{Homework 4 MARE}
\label{sec-1}
\subsection{(Chapter 6 from Practical Malware Analysis}
\label{sec-1-1}



\section{Lab 06-01}
\label{sec-2}

\subsection{Question 1}
\label{sec-2-1}

The main function calls in main is \texttt{InternetGetConnectedState}.
If the host can connect to the internet, it calls an internal function
\texttt{sub 0x40105F}, which proceeds to run this massive code construct.

\subsection{Question 2}
\label{sec-2-2}
This massive code construct at \texttt{sub 0x40105F} that seems to be a Microsoft
Visual C++ runtime Library call based on the it's complexity and numerous
internal functions. There are lots of dynamic memory calls related to
\texttt{malloc} or \texttt{heaps}.  
\subsection{Question 3}
\label{sec-2-3}
Performing dynamic analysis shows that the program simply prints "Connected
to internet" if the can and "Error 1.1 No Internet" depending on the outcome
of \texttt{InternetGetConnectedState}. One might conclude then, the funcition at
\texttt{sub 0x40105F} is something along the lines of a \texttt{printf} statement from the
system library.

\section{Lab 06-02}
\label{sec-3}
\subsection{Question 1}
\label{sec-3-1}
The first subroutine makes a call to check for internet connection.    
\subsection{Question 2}
\label{sec-3-2}
In the subroutine located at \texttt{0x40117F}, we see a call to call to another
subroutine from a seemingly static library, \texttt{\_stbuf} which seems to allocate
some dynamic buffer. Then, there seems to be some file handling and file
pointer parsing. Finally, the \texttt{\_frbuf} seems to free the dynamically
allocated buffer.  Looks like an unlabeled \texttt{printf()} call again.
\subsection{Question 3}
\label{sec-3-3}
The second call is pretty obvious. It tries to connect to an external URL,
\texttt{practicalmalwareanalysis}, read 200h bytes and then parse it for some kind
of command. Looks like it's parsing HTML code. There is a series of checks
that checks for the string "<!--".  The argument is saved in register AL.
\subsection{Question 4}
\label{sec-3-4}
A series of if conditionals are used.
\subsection{Question 5}
\label{sec-3-5}
Looking at the imports, we see that there are strings like \texttt{InternetReadFile}
and \texttt{Success Internet Connection}. There's also a URL pointed to the
\texttt{practicalmalwareanalysis.com}. There's also a string for \texttt{Internet Explorer
7.5}. We can probably check for any traffic to this URL with this browser. 
\subsection{Question 6}
\label{sec-3-6}
Looks like the program tries to connect to the internet, connect get some
HTML page. Parse a comment for a command and then sleep for a minute.
\section{Lab 06-03}
\label{sec-4}
\subsection{Question 1}
\label{sec-4-1}
Compared to the function call from Lab 06-02, there is a new subroutine that
plays with registers, \texttt{sub\_401130}. It tries to open a register of windowsversion and set
the key to "malware" and the data to point to some file at \texttt{C:\textbackslash{}Temp\textbackslash{}cc.exe}.
There is also a jump table which seems take a file and drop it into that
specific directory. Seems highly suspicious.
\subsection{Question 2}
\label{sec-4-2}
The new function takes a pointer to a string containing a filename. Also an
argument, which is probably the command parsed from the HTML source from
before.
\subsection{Question 3}
\label{sec-4-3}
The construct is a jump table. Probably a switch. Can do things like create
new directory, copy file, delete file, set registers or sleep or exit. 
\begin{itemize}
\item New dir
\item Copy
\item delete
\item set regs
\item sleep
\item exit (error parsing)
\end{itemize}
\subsection{Question 4}
\label{sec-4-4}
This function can change register values. Sets the \texttt{malware} value in \texttt{CurrentVersion\textbackslash{}Run} to
\texttt{C:\textbackslash{}Temp\textbackslash{}cc.exe}.
\subsection{Question 5}
\label{sec-4-5}
Seems to open a link to \texttt{http://practicalmalwareanalysis.com/cc.htm} and also
creates a file at \texttt{C:\textbackslash{}Temp\textbackslash{}cc.exe}. 
\subsection{Question 6}
\label{sec-4-6}
Purpose seems to grab a file from the internet and copy it into to known
location for future use. Allows persistence for backdoor commands.
\section{Lab 06-04}
\label{sec-5}
\subsection{Compare the \texttt{main} method with that of 06-02, 06-03.}
\label{sec-5-1}
Well, this one seems to be built of 06-03 with a new loop mechanism. 

\subsection{What new code has added to main()?}
\label{sec-5-2}
There is now a loop function for attempting to parse the command. This loop
iterates 1440 times after the internet connection passes.   
\subsection{How is this HTML parse function different from the previous labs?}
\label{sec-5-3}
This parse function is a little different in the user agent has a \%d. There
is a call to print string which seems to be the loop iteration of the
program prefixed with \texttt{pma}.
\subsection{How long will the program run?}
\label{sec-5-4}
Runs for 1440 iterations with each iteration lasting a minute. This means the
program runs for an entire day.
\subsection{Network-based indicators?}
\label{sec-5-5}
Network based indicators is a repeated request to
\texttt{http://practicalmalwareanalysis.com/cc.htm} using the \texttt{Internet Explorer
7.5} client agent.
\subsection{What is the purpose of this malware?}
\label{sec-5-6}
Perform realtime coordination of a potential botnet for a single day.
% Emacs 25.2.2 (Org mode 8.2.10)
\end{document}
