% Created 2020-04-19 Sun 15:59
\documentclass[11pt]{article}
\usepackage[utf8]{inputenc}
\usepackage[T1]{fontenc}
\usepackage{fixltx2e}
\usepackage{graphicx}
\usepackage{longtable}
\usepackage{float}
\usepackage{wrapfig}
\usepackage{rotating}
\usepackage[normalem]{ulem}
\usepackage{amsmath}
\usepackage{textcomp}
\usepackage{marvosym}
\usepackage{wasysym}
\usepackage{amssymb}
\usepackage{hyperref}
\tolerance=1000
\author{dj$_{\text{dvorak}}$}
\date{\today}
\title{hw5}
\hypersetup{
  pdfkeywords={},
  pdfsubject={},
  pdfcreator={Emacs 25.2.2 (Org mode 8.2.10)}}
\begin{document}

\maketitle
\tableofcontents

\section{Lab 7 Analyzing Malicious Windows Programs}
\label{sec-1}


\section{Lab 07-01}
\label{sec-2}
\subsection{Achieving Persistence}
\label{sec-2-1}
This file achieves persistence by creating a service called \texttt{MalService}. 
\subsection{Mutex Purpose}
\label{sec-2-2}
This program uses a mutex called \texttt{HGL345} to make sure their is only one
instance of the original program running.
\subsection{Good Host-based signature}
\label{sec-2-3}
You can check for the creation of a process called \texttt{Malware Service} or the
name of the strange mutex \texttt{HGL345}. The service is automatically tied to the
binary's path and filename.
\subsection{Good Network-based signature}
\label{sec-2-4}
Any traffic to the site \verb~practicalmalwareanalysis.com~ with \verb~Internet Explorer 8.0~. 
\subsection{Purpose}
\label{sec-2-5}
Seems to be a DDOS attcker? 20 threads are spawned starting in the year 2100. 
\subsection{Duration of Execution}
\label{sec-2-6}
Seems each thread will sleep for \texttt{0xffffff} milliseconds. But the attacks do
not stop otherwise.
\section{Lab 07-02}
\label{sec-3}
\subsection{Persistence}
\label{sec-3-1}
It doesn't look like this program is persistent.      
\subsection{Purpose}
\label{sec-3-2}
The purpose seems to be open an ad with Internet Explorer, then exits. This
software does so by using the OLE \texttt{CoCreateInstance} call which uses COM
objects to open programs. The CLSID key is the key to opening IE. 

\texttt{http://practicalmalwareanalysis.com/ad.html}.  
\subsection{Duration of Execution}
\label{sec-3-3}
Seems to run and then quits if anything fails (after the ad is opened).
\section{Lab 07-03}
\label{sec-4}
\subsection{How Does this Program Achieve Persistence}
\label{sec-4-1}
The original exe file copies the dll file into a fake windows dll under the
path of \texttt{C:\textbackslash{}windows\textbackslash{}system32\textbackslash{}kerne132.dll} . Note the 1 instead of the l.
\subsection{Two good host-based indicators}
\label{sec-4-2}
Well, there is this string that says \texttt{THIS WILL DESTROY YOUR COMPUTERT}.
Also, we see that mutex is created by the dll with the name \texttt{SADFHUHF}. An
IP address is listed here 127.26.152.13 (seemingly changed for safety
purposes). Additionally, it looks like it is listening for commands such as
"exec", "sleep" and "q" from this external IP. 
\subsection{The purpose}
\label{sec-4-3}
The purpose is to inject their bogus .dll file into scan the possibly change
any system calls to point to their bogus kerne132.dll \ldots{} This also runs some
back door serivce which allows for the arbitrary execution of commands and
processes! Looks deadly.            
\subsection{How to remove this malware after Installation}
\label{sec-4-4}
Well, the damage is pretty bad since it modifies all system files. Better to
just restore a backup image! 
% Emacs 25.2.2 (Org mode 8.2.10)
\end{document}
