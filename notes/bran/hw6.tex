% Created 2020-04-26 Sun 13:18
% Intended LaTeX compiler: pdflatex
\documentclass[11pt]{article}
\usepackage[utf8]{inputenc}
\usepackage[T1]{fontenc}
\usepackage{graphicx}
\usepackage{grffile}
\usepackage{longtable}
\usepackage{wrapfig}
\usepackage{rotating}
\usepackage[normalem]{ulem}
\usepackage{amsmath}
\usepackage{textcomp}
\usepackage{amssymb}
\usepackage{capt-of}
\usepackage{hyperref}
\usepackage{minted}
\author{dj\textsubscript{dvorak}}
\date{\today}
\title{MARE HW 6}
\hypersetup{
 pdfauthor={dj\textsubscript{dvorak}},
 pdftitle={MARE HW 6},
 pdfkeywords={},
 pdfsubject={},
 pdfcreator={Emacs 25.2.2 (Org mode 9.3.6)}, 
 pdflang={English}}
\begin{document}

\maketitle
\tableofcontents


\section{Lab 9-1}
\label{sec:org259ed1f}

\subsection{How do you install?}
\label{sec:org587a113}
It seems to want an argument \texttt{-in "password"} which will install itself to the System32 folder. A
A registry is created along with a service, presumably for persistence.

\subsection{What are the command line arguments?}
\label{sec:org07d934c}
Well, the main file seems to include two arguments 

It looks like there are two arguments in the main. It will check for the args
\texttt{-re}, then \texttt{-in}. If the \texttt{-in} is passed, it also checks for \texttt{-c} and \texttt{-cc = flags.
It turns out that =-in} will install, \texttt{-re} will uninstall and \texttt{-c} will pass in 4 arguments
to update the configuratino and \texttt{-cc} will print the configuration. 

\subsection{Can we patch to remove the password?}
\label{sec:orgbff4e0c}

Sure. Since we know where the desired the function is (\texttt{0x0000402b61}), we can modify the jump at 
the beginning of the program to make it unconditional.

\subsection{Host based-indicators}
\label{sec:orgbe03220}
Modifies the registers in \texttt{\textbackslash{}\textbackslash{}SOFTWARE\textbackslash{}Microsoft\textbackslash{}XPS}. 
Also, a copy of itself in system32 folder
\subsection{The Pontential Remote Actions}
\label{sec:org87fe65a}
Through the networked commands via HTTP requests, we can 

\begin{itemize}
\item Download
\item Sleep
\item Do nothing
\item Upload
\item Run CMD (shell scripts)
\end{itemize}

\subsection{Any useful network based signatures?}
\label{sec:org8792afd}
Well, it sends network requests to and from \texttt{https://practicalmalwareanalysis.com}. Uses GET request 
with HTTP 1.0. Seems to ask for different strings each time though.

\section{Lab 9-2}
\label{sec:org3657677}

\subsection{What strings do you statically see in the binary?}
\label{sec:orgf42d778}
The usual KERNEL32.dll, WSAsockets, Create and Terminate Processes. But also, unexpectedly
we see some function calls to what appears to be getting and setting environmental variables.
Nothing special.

\subsection{Run the binary?}
\label{sec:org2874018}

I will not run the binary

\subsection{How to run this malicious payload?}
\label{sec:orgf00c4d6}

The main exe takes a file argument and loads it through \texttt{GetModuleFileName}. 

\subsection{Examining \texttt{0x000401133}}
\label{sec:org144ae97}

So this occurs near the beginning of the \texttt{main()} and it loads a bunch of what appears to be
ASCII characters. It looks like \texttt{1qaz2wsx3edc\textbackslash{}0ocl.exe}. Later on,
we realize this is two different strings: one for encoding and the other
is simply the name of the url.  

\subsection{Examining \texttt{0x00401089}}
\label{sec:org924948a}
Well, this seems to be a function call that takes two args, a char* and a integer. There is a 
call to \texttt{strlen()} and then a loop that calls an xor function. Just guessing but it might be an
encryption algorithm. Actually, the argument seems to be the array from \texttt{0x00401133}.

\subsection{Domain name}
\label{sec:orge505dcf}
This name is \texttt{practicalmalwareanalysis.com}

\subsection{What encoding routine is being used?}
\label{sec:org9f85ec7}
Probably a caesar shiftp with the XOR procedure.

\subsection{On \texttt{CreateProcessA} at \texttt{0x0040106E}}
\label{sec:org4b8170d}
This calls a command prompt and allows the execution of arbitrary commands. It is probably connected 
to the host domain name.
\end{document}
