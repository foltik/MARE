% Created 2020-04-23 Thu 03:16
% Intended LaTeX compiler: pdflatex
\documentclass[11pt]{article}
\usepackage[utf8]{inputenc}
\usepackage[T1]{fontenc}
\usepackage{graphicx}
\usepackage{grffile}
\usepackage{longtable}
\usepackage{wrapfig}
\usepackage{rotating}
\usepackage[normalem]{ulem}
\usepackage{amsmath}
\usepackage{textcomp}
\usepackage{amssymb}
\usepackage{capt-of}
\usepackage{hyperref}
\author{Andrew Park}
\date{\today}
\title{}
\hypersetup{
 pdfauthor={Andrew Park},
 pdftitle={},
 pdfkeywords={},
 pdfsubject={},
 pdfcreator={Emacs 26.3 (Org mode 9.1.9)}, 
 pdflang={English}}
\begin{document}

\tableofcontents

\section{Homework}
\label{sec:orgb3a89a2}
\subsection{HW6}
\label{sec:org78563ec}
Done using Ghidra
\subsubsection{Phase 1}
\label{sec:orga4623fa}
Phase 1 is simply a stirng comparison. By looking at the string, we can see the solution is \texttt{"Public speaking is very easy."}
\begin{center}
\includegraphics[width=.9\linewidth]{./phase1graph.png}
\end{center}
\subsubsection{Phase 2}
\label{sec:org0fae86c}
\begin{center}
\includegraphics[width=.9\linewidth]{./phase2graph.png}
\end{center}

Taking a look at the phase shows that 6 numbers are read from the user. It can be seen the the first number must be a 1. Further down the program, there is a loop that iterates through the rest of the numbers.

\begin{center}
\includegraphics[width=.9\linewidth]{./phase2loop.png}
\end{center}

The segment that is looped checks if the next number is equal to its index+1, multiplied by the last index.

\begin{center}
\includegraphics[width=.9\linewidth]{./phase2cond.png}
\end{center}

This means our solution is \texttt{1 2 6 24 120 720}
\subsubsection{Phase 3}
\label{sec:orga0c8ffe}
This phase is a simple 2 numbers with a character.

\begin{center}
\includegraphics[width=.9\linewidth]{./phase3init.png}
\end{center}

There is a jump table where the second number is compared with a required value, and loads an associated character to be compared.

\begin{center}
\includegraphics[width=.9\linewidth]{./phase3case.png}
\end{center}

Using the first case, we get the solution \texttt{0 q 777}
\subsubsection{Phase 4}
\label{sec:org00f0c3b}
The phase reads 1 number which must be greater than 0. This number is then passed into a function, and explodes if the returned value is not 55

\begin{center}
\includegraphics[width=.9\linewidth]{./phase4graph.png}
\end{center}

The function can be described as:

\begin{verbatim}
def func4(int n):
    if n<=1:
	return 1
    else:
	return func4(n-1) + func4(n-2)
\end{verbatim}

func4 is clearly the fibonacci numbers.

Thus, the solution to this phase is \texttt{9}
\subsubsection{Phase 5}
\label{sec:org490b1c2}
This phase takes in a string input of length 6

This string is then translated using a table and then compared with the string \texttt{"giants"}

\begin{center}
\includegraphics[width=.9\linewidth]{./phase5graph.png}
\end{center}

The function takes the lower 4 bits of each character, indexing it with the said table to get the resulting character.

\begin{center}
\includegraphics[width=.9\linewidth]{./phase5table.png}
\end{center}

Using the table, we can find the required byte sequences to produce \texttt{"giants"}. The sequence is \texttt{1111, 0000 0101, 1011, 1101, 0001}. We can then use the byte sequence to generate a valid solution.

One possible solution is \texttt{"OPEKMA"}
\subsubsection{Phase 6}
\label{sec:org6d4ee93}
Phase 6 starts by reading in 6 numbers.

The function then checks to make sure all 6 numbers are < 7

\begin{center}
\includegraphics[width=.9\linewidth]{./phase6checkinit.png}
\end{center}

It then checks to make sure all 6 numbers are unique

\begin{center}
\includegraphics[width=.9\linewidth]{./phase6unique.png}
\end{center}

We see that there are structures in the data segment of the program. 

Following the memory address, we find 6 nodes in data. Setting their data types as nodeStruct allows us to view their contents.

\begin{center}
\includegraphics[width=.9\linewidth]{./phase6nodes.png}
\end{center}

Inspecting these values, we see that it is a linked list. Sorting them in decreasing order yields the solution \texttt{4 2 6 3 1 5}

This marks the end of the 6 phases of Bomb Lab. However, there is another phase that we can reach which is hidden unless we look at the underlying code

\subsubsection{Secret Phase}
\label{sec:org70d3b14}
To get to the secret phase, we must first look at the function \texttt{phase\_defused}. 

\begin{center}
\includegraphics[width=.9\linewidth]{./phasedefused.png}
\end{center}

Inside \texttt{phase\_defused}, we see that there is a call to \texttt{secret\_phase}, though it can not normally be reached. To reach it, we must append a string to the end of the solution of phase 4. Taking a look at the string, we can append \texttt{"austinpowers"} to the end of the solution of phase 4 to reach this secret phase

\begin{center}
\includegraphics[width=.9\linewidth]{./secretphase.png}
\end{center}

Taking a look at the secret phase, we see that initially it checks if the input is less than or equal to 1001.

Afterwards, there is a function call \texttt{fun7}, which is then compared and explodes if the value is not 7.

A pointer to segments in data can be found in the initial call to \texttt{fun7}. 

\begin{center}
\includegraphics[width=.9\linewidth]{./secretdata.png}
\end{center}

Taking a look at the data, we can clearly see a binary tree.

If the passed node is zero, the function returns -1. If the node's value is less than the input, it returns double the result of recursing the left child, and the same input. Otherwise, it returns twice the result+1 of recursing the right child, and the same input

To solve this, a path must be found where the resultant is 7

\begin{center}
\includegraphics[width=.9\linewidth]{./fun7.png}
\end{center}

Tracing the call structure, we find the solution \texttt{1001}, which defuses the secret phase.

This marks the end of Bomb Lab
\end{document}
