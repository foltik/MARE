% Created 2020-04-26 Sun 15:39
% Intended LaTeX compiler: pdflatex
\documentclass[11pt]{article}
\usepackage[utf8]{inputenc}
\usepackage[T1]{fontenc}
\usepackage{graphicx}
\usepackage{grffile}
\usepackage{longtable}
\usepackage{wrapfig}
\usepackage{rotating}
\usepackage[normalem]{ulem}
\usepackage{amsmath}
\usepackage{textcomp}
\usepackage{amssymb}
\usepackage{capt-of}
\usepackage{hyperref}
\author{Andrew Park}
\date{\today}
\title{}
\hypersetup{
 pdfauthor={Andrew Park},
 pdftitle={},
 pdfkeywords={},
 pdfsubject={},
 pdfcreator={Emacs 26.3 (Org mode 9.1.9)}, 
 pdflang={English}}
\begin{document}

\tableofcontents

\section{Homework}
\label{sec:org92d9779}
\subsection{HW7}
\label{sec:org32b18c6}
\subsubsection{Lab 9-1}
\label{sec:org2dcdea6}
\begin{enumerate}
\item Question 1
\label{sec:org7d58fe6}
To get the malware to install, we need to reach \texttt{0x00402600}. Within this function, there are function calls to \texttt{OpenSCManagerA}, \texttt{ChangeServiceConfigA}, \texttt{CreateServiceA}, \texttt{CopyFileA}, and registry modifications. 

To get to the install function, we would need to run this malware with 3 arguments. We need a password as one of these arguments along with \texttt{"-in"} as the first argument. To decipher this password, we can take a look at \texttt{0x00402510}. The password must be 4 characters long. After analyzing the funciton, we see that the passcode is \texttt{"abcd"}. 

\begin{center}
\includegraphics[width=.9\linewidth]{./lab91a.png}
\end{center}

We can also patch \texttt{0x00402B38} by changing \texttt{jnz} to \texttt{jz} to bypass any password check.

We can install the malware by executing it with the arguments \texttt{"Lab09-01.exe -in abcd"}.
\item Question 2
\label{sec:org870bbe6}
There are 4 command-line options for the program.

\begin{enumerate}
\item \texttt{"-in"}: installs

\item \texttt{"-re"}: uninstalls

\item \texttt{"-cc"}: prints our registry

\item \texttt{"-c"}: sets registry value
\end{enumerate}

\begin{center}
\includegraphics[width=.9\linewidth]{./lab91b.png}
\end{center}

Analyzing the function, we can find what the password to the installer, \texttt{"abcd"}.
\item Question 3
\label{sec:org902073d}
To Patch the program so that it doesnt require a password, we need to patch 0x00402B38 from \texttt{jnz} to \texttt{jz}. 

\begin{center}
\includegraphics[width=.9\linewidth]{./lab91c.png}
\end{center}

Changing \texttt{75} to \texttt{74} will change the instruction from \texttt{jnz} to \texttt{jz}.

\item Question 4
\label{sec:org4c14715}
\begin{center}
\includegraphics[width=.9\linewidth]{./lab91d1.png}
\end{center}

One possible way of detecting the malware is by checking if any registry values were added. These registry keys are added to create persistance.

\begin{center}
\includegraphics[width=.9\linewidth]{./lab91d2.png}
\end{center}

We can check for these registry keys as a way of detecting whether a computer has been infected or not
\item Question 5
\label{sec:org4e64f8a}
\begin{center}
\includegraphics[width=.9\linewidth]{./lab91e.png}
\end{center}
Taking a look at 0x00402020, we can see that there are multiple different tasks the malware does

\begin{enumerate}
\item Sleep

\item Upload

\item Download

\item Execute

\item Do nothing
\end{enumerate}
\item Question 6
\label{sec:orgad483d7}
\begin{center}
\includegraphics[width=.9\linewidth]{./lab91f.png}
\end{center}
We can see in the strings, there is a website stored in the program, \texttt{"http://www.practicalmalwareanalysis.com"}. Using wireshark, we see that the malware is trying to receive commands from the website
\end{enumerate}
\subsubsection{Lab 9-2}
\label{sec:org380ae6f}
\end{document}
